\subsection{Bewerten der Lösungen gegenüber den Anforderungen}

Das Ziel dieser Phase ist es, die entwickelten Lösungen (Papierprototyp und \gls{browser}-Erweiterung) hinsichtlich ihrer Erfüllung der festgelegten Anforderungen zu bewerten. 

\textbf{Methoden zur Bewertung:}
\begin{itemize}
    \item \textbf{Usability-Tests:} Durchführung von Tests mit echten Nutzern, um die Benutzerfreundlichkeit und die intuitive Nutzung der hierarchischen Tab-Verwaltung zu evaluieren.
    \item \textbf{Feedback-Sitzungen:} Regelmäßige Sitzungen mit Stakeholdern und Nutzern, um deren Erfahrungen und Verbesserungsvorschläge zu diskutieren.
    \item \textbf{Technische Performance-Analyse:} Überprüfung der Performance der \gls{browser}-Erweiterung, um sicherzustellen, dass sie die Leistung des \gls{browser}s nicht negativ beeinflusst.
\end{itemize}

\textbf{Ergebnisse der Bewertung:}
\begin{itemize}
    \item \textbf{Usability-Tests:}
    \begin{itemize}
        \item \textbf{Positive Rückmeldungen:} Nutzer fanden die hierarchische Struktur hilfreich beim Organisieren von \gls{tab}s zu bestimmten Themen. Die schnelle Navigation zwischen Tabs wurde als besonders nützlich empfunden.
        \item \textbf{Verbesserungspotential:} Einige Nutzer wünschten sich eine Suchfunktion und die Möglichkeit, Tabs innerhalb der hierarchischen Struktur zu verschieben.
    \end{itemize}
    \item \textbf{Feedback-Sitzungen:}
    \begin{itemize}
        \item \textbf{Nutzeranregungen:} Teilnehmer schlugen vor, die Benutzeroberfläche weiter zu vereinfachen und die Suchfunktion als Priorität zu implementieren.
        \item \textbf{Stakeholder-Kommentare:} Stakeholder betonten die Wichtigkeit der Performance und die Notwendigkeit, dass die Erweiterung stabil und zuverlässig läuft.
    \end{itemize}
    \item \textbf{Technische Performance-Analyse:}
    \begin{itemize}
        \item \textbf{Performance-Ergebnisse:} Die Erweiterung beeinflusst die Leistung des \gls{browser}s nicht merklich, selbst bei vielen geöffneten \gls{tab}s.
        \item \textbf{Optimierungsmöglichkeiten:} Weiterer Code-Optimierungen können durchgeführt werden, um die Effizienz der Erweiterung zu maximieren.
    \end{itemize}
\end{itemize}

\textbf{Zusammenfassung und nächste Schritte:}
Basierend auf der Bewertung der Lösungen gegenüber den Anforderungen wurden mehrere Verbesserungspotentiale identifiziert. Die Implementierung einer Suchfunktion und die Möglichkeit, Tabs innerhalb der hierarchischen Struktur zu verschieben, wurden als Prioritäten erkannt.

\begin{itemize}
    \item \textbf{Kurzfristige Maßnahmen:}
    \begin{itemize}
        \item Entwicklung und Integration einer Suchfunktion.
        \item Implementierung der Tab-Verschiebungsfunktion innerhalb der hierarchischen Struktur.
        \item Durchführung weiterer Usability-Tests nach der Implementierung der neuen Funktionen.
    \end{itemize}
    \item \textbf{Langfristige Maßnahmen:}
    \begin{itemize}
        \item Kontinuierliche Optimierung der Performance und Zuverlässigkeit der \gls{browser}-Erweiterung.
        \item Regelmäßige Feedback-Sitzungen zur Ermittlung neuer Nutzeranforderungen und Verbesserungsvorschläge.
    \end{itemize}
\end{itemize}
