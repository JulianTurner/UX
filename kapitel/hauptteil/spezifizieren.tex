\subsection{Festlegen der Benutzerbelange und der Erfordernisse}

\textbf{Ziel:} Das Ziel dieses Schrittes ist es, die Benutzerbelange und Erfordernisse für ein Browser-Plugin zur Tab-Verwaltung mit einer Baumstruktur zu spezifizieren. Diese Anforderungen dienen als Grundlage für das Design und die Entwicklung des Plugins.

\textbf{Benutzerbelange und Erfordernisse:}

\begin{enumerate}
    \item \textbf{Übersichtlichkeit und Organisation:}
    \begin{itemize}
        \item Nutzer müssen in der Lage sein, eine große Anzahl von Tabs effizient zu verwalten.
        \item Eine Baumstruktur soll eine hierarchische Organisation der Tabs ermöglichen.
        \item Tabs sollten einfach in der Baumstruktur verschoben, gruppiert und verschachtelt werden können.
    \end{itemize}
    
    \item \textbf{Benutzerfreundlichkeit:}
    \begin{itemize}
        \item Das Plugin muss eine intuitive und leicht verständliche Benutzeroberfläche haben.
        \item Drag-and-Drop-Funktionalität zum Verschieben und Organisieren von Tabs.
        \item Schneller Zugriff auf häufig genutzte Funktionen und Einstellungen.
    \end{itemize}
    
    \item \textbf{Leistungsfähigkeit:}
    \begin{itemize}
        \item Das Plugin soll auch bei einer großen Anzahl von Tabs flüssig und schnell arbeiten.
        \item Minimale Auswirkungen auf die Browserleistung.
    \end{itemize}
    
    \item \textbf{Such- und Filterfunktionen:}
    \begin{itemize}
        \item Möglichkeit, Tabs nach Titel, URL oder Inhalt zu durchsuchen.
        \item Filteroptionen, um Tabs nach verschiedenen Kriterien zu sortieren (z.B. nach Erstellungsdatum, Domain).
    \end{itemize}
    
    \item \textbf{Anpassbarkeit:}
    \begin{itemize}
        \item Nutzer sollen die Möglichkeit haben, das Erscheinungsbild und die Funktionen des Plugins an ihre Bedürfnisse anzupassen.
        \item Unterstützung für verschiedene Themes und Layout-Optionen.
    \end{itemize}
    
    \item \textbf{Synchronisation:}
    \begin{itemize}
        \item Möglichkeit zur Synchronisation der Tab-Baumstruktur über mehrere Geräte hinweg.
    \end{itemize}
\end{enumerate}

\textbf{Methoden zur Ermittlung der Anforderungen:}

\begin{itemize}
    \item \textbf{Benutzerbefragungen:} Durchführen von Umfragen und Interviews mit den Nutzern, um ihre spezifischen Bedürfnisse und Wünsche zu erfassen.
    \item \textbf{Wettbewerbsanalyse:} Analyse bestehender Tab-Verwaltungs-Plugins, um deren Stärken und Schwächen zu identifizieren und daraus Anforderungen abzuleiten.
    \item \textbf{Nutzungsszenarien und Use Cases:} Erstellen von detaillierten Nutzungsszenarien und Use Cases, um die verschiedenen Anwendungsfälle des Plugins zu dokumentieren.
    \item \textbf{Prototyping und Benutzer-Feedback:} Entwicklung von Prototypen des Plugins und Einholen von Feedback der Nutzer, um die Anforderungen zu validieren und zu verfeinern.
\end{itemize}

\textbf{Dokumentation der Anforderungen:}

Die Anforderungen werden in einem Anforderungsdokument festgehalten, das die Grundlage für das Design und die Entwicklung des Plugins bildet. Es wird regelmäßig überprüft und aktualisiert, um sicherzustellen, dass es den aktuellen Bedürfnissen der Nutzer entspricht.


\textbf{Ziel:}

% - Leistung im Bezug auf betriebliche & finanzielle Ziele
% - Vorschriften & Gesetze
% - Zusammenarbeit mit anderen Systemen
% - Durchführbarkeit von Betrieb & Wartung (z.B. SLA, Alternativen)

% - Verwenden des Wissens aus Ergonomie, Psychologie, ...
% - Konkretisierung von Gestaltungslösungen mit Pseudoimplementierungen (z.B. Prototypen, Mockups)
% - Wiederholung des Prozesses bis Zufriedenheit

