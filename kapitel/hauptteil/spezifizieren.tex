\subsection{Festlegen der Benutzerbelange und der Erfordernisse}

\textbf{Ziel:} 
Das Ziel dieser Phase ist es, die spezifischen Anforderungen und Belange der Benutzer des \gls{browser}s zu definieren und zu dokumentieren. 
Dies geschieht auf Basis der Erkenntnisse aus der Kontextanalyse und dient als Grundlage für die Entwicklung benutzerzentrierter Lösungen, wie in diesem Fall eine Erweiterung des \gls{browser}s zur Verwaltung von \gls{browser}-\gls{tab}s mit einer Baumstruktur.

\textbf{Anforderungsarten:}
\begin{itemize}
    \item \textbf{Funktionale Anforderungen:}
    \begin{itemize}
        \item \textbf{Effiziente Tab-Verwaltung:} Der \gls{browser} muss eine hierarchische Struktur zur Verwaltung von \gls{tab}s unterstützen. Diese Struktur soll es Benutzern wie Robert und Lena ermöglichen, alle Tabs zu einem bestimmten Thema zu gruppieren und bei Bedarf auf einmal zu schließen.
        \item \textbf{Suchfunktion:} Der \gls{browser} muss eine schnelle und effiziente Suchfunktion bieten, die es ermöglicht, bestimmte \gls{tab}s schnell zu finden und zu öffnen.
        \item \textbf{Tabs verschieben:} Benutzer müssen die Möglichkeit haben, \gls{tab}s innerhalb der hierarchischen Struktur einfach zu verschieben, um die Hierarchie an ihre Bedürfnisse anzupassen.
        \item \textbf{Intuitive Benutzeroberfläche:} Die Oberfläche sollte leicht verständlich und benutzerfreundlich sein, um Benutzern wie Lena das Organisieren und Wiederfinden von \gls{tab}s zu erleichtern.
    \end{itemize}
    \item \textbf{Nicht-funktionale Anforderungen:}
    \begin{itemize}
        \item \textbf{Performance:} Der \gls{browser} muss eine hohe Performance bieten, um auch bei vielen geöffneten \gls{tab}s schnell zu bleiben. Die Erweiterung darf die Performance des \gls{browser}s nicht einschränken.
        \item \textbf{Zuverlässigkeit:} Der \gls{browser} soll stabil laufen und nicht abstürzen, selbst bei intensiver Nutzung.
        \item \textbf{Zugänglichkeit:} Die Funktionen des \gls{browser}s sollten auch für Nutzer mit speziellen Bedürfnissen zugänglich sein.
    \end{itemize}
\end{itemize}

\textbf{Methoden zur Erhebung und Validierung der Anforderungen:}
\begin{itemize}
    \item \textbf{Workshops und Brainstorming-Sitzungen:} Durchführung von Sitzungen mit Stakeholdern und Nutzern, um die Anforderungen zu diskutieren und zu validieren.
    \item \textbf{User Stories:} Erstellung von \gls{userStory}, um spezifische Nutzungsszenarien zu beschreiben.
\end{itemize}

\textbf{Dokumentation:}
Die definierten Anforderungen werden in einem Anforderungsdokument festgehalten, das für alle Projektbeteiligten zugänglich ist. 
Dieses Dokument enthält detaillierte Beschreibungen der funktionalen und nicht-funktionalen Anforderungen sowie die \gls{userStory}.

\textbf{Erstellte \gls{userStory}:}
\begin{itemize}
    \item \textbf{User Story 1:} Als Robert möchte ich eine effiziente Möglichkeit, über 100 \gls{tab}s zu verwalten, damit ich schnell die benötigten Informationen finde.
    \item \textbf{User Story 2:} Als Lena möchte ich eine intuitive Benutzeroberfläche, die mir hilft, meine Recherchen zu organisieren und wiederzufinden.
    \item \textbf{User Story 3:} Als Anton möchte ich eine leistungsstarke Suchfunktion, um spezifische \gls{tab}s schnell zu finden.
    \item \textbf{User Story 4:} Als Benutzer möchte ich die Möglichkeit haben, \gls{tab}s innerhalb der hierarchischen Struktur zu verschieben, um meine \gls{tab}-Hierarchie flexibel anzupassen.
    \item \textbf{User Story 5:} Als Robert oder Lena möchte ich die Möglichkeit haben, alle \gls{tab}s zu einem bestimmten Thema auf einmal zu schließen, um meine \gls{browser}-Sitzung effizient zu verwalten.
    \item \textbf{User Story 6:} Als Anton möchte ich die Möglichkeit haben, schnell zwischen verschiedenen \gls{tab}s zu wechseln, um Informationen schnell zu vergleichen.
    \item \textbf{User Story 7:} Als Benutzer möchte ich sicherstellen, dass die Performance des \gls{browser}s durch die Erweiterung nicht eingeschränkt wird, damit der \gls{browser} auch bei vielen geöffneten \gls{tab} schnell bleibt.
\end{itemize}