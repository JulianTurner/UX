\subsection{Verstehen und Spezifizieren des Nutzungskontextes}

Das Ziel dieser Phase ist es, ein tiefes Verständnis dafür zu entwickeln, wer die Nutzer des \gls{browser} sind, in welchem Kontext sie den \gls{browser} verwenden und welche spezifischen Anforderungen und Bedürfnisse sie haben. 
Um diese Informationen zu sammeln, wurden verschiedene Methoden eingesetzt, darunter Umfragen und Interviews, die durch einen Fragebogen gestützt wurden.

\textbf{Fragebogen:}
\begin{itemize}
    \item Wie alt sind Sie?
    \item In welchem Kontext nutzen Sie den \gls{browser}? (z.B. Arbeit, Studium, Freizeit)
    \item Wie viele \gls{tab}s haben Sie durchschnittlich geöffnet?
    \item Welche Funktionen des \gls{browser}s nutzen Sie am häufigsten?
    \item Welche Herausforderungen oder Frustrationen haben Sie bei der Nutzung des \gls{browser}s?
    % \item Welche Features würden Sie sich im \gls{browser} wünschen?
    \item Welche Hardware und andere Software nutzen Sie in Verbindung mit dem \gls{browser}?
    \item Wie sieht Ihre physische und soziale Umgebung bei der Nutzung des \gls{browser}s aus? (z.B. Arbeitsplatz, Home-Office, unterwegs)
\end{itemize}

\textbf{Bestätigung der Ergebnisse:}
Die gesammelten Informationen wurden durch Feedback-Sitzungen mit den Nutzern validiert. 
Daraufhin wurden Personas definiert, die die Nutzergruppen repräsentieren. 
Dies stellte sicher, dass die Anforderungen und Bedürfnisse korrekt erfasst wurden.

\textbf{Personas:}
\begin{enumerate}
    \item \textbf{Robert}:
    \begin{itemize}
        \item Alter: 35 Jahre
        \item Beruf: Softwareentwickler
        \item Arbeitsumfeld: Arbeitet von zu Hause aus
        \item Nutzungsverhalten: Nutzt den \gls{browser} hauptsächlich zur Recherche im Projekt- und Arbeitsumfeld
        \item Spezifische Bedürfnisse: Hat oft über 100 \gls{tab}s geöffnet, benötigt eine effiziente \gls{tab}-Verwaltung und schnelle Suchfunktionen innerhalb der \gls{tab}s
        \item Herausforderungen: Verliert oft den Überblick über geöffnete \gls{tab}s
        \item Digitale Umgebung: Nutzt einen leistungsstarken Desktop-Computer mit mehreren Monitoren, häufig verwendete Software: \ac{IDE}, Dokumentations-Tools
        \item Physische und soziale Umgebung: Home-Office, arbeitet oft allein, gelegentlich in virtuellen Teamsitzungen
    \end{itemize}
    \item \textbf{Lena}:
    \begin{itemize}
        \item Alter: 25 Jahre
        \item Beruf: Studentin
        \item Nutzungsverhalten: Nutzt den \gls{browser} für Recherche und zum Schreiben von Hausarbeiten
        \item Spezifische Bedürfnisse: Nutzt ca. 50 \gls{tab}s, benötigt gute Bookmarking-Funktionen und eine intuitive Benutzeroberfläche, die das Organisieren und Wiederfinden von \gls{tab}s erleichtert
        \item Herausforderungen: Schwierigkeiten beim Organisieren von Recherchematerial
        \item Digitale Umgebung: Laptop mit Standard-Software (z.B. Textverarbeitungsprogramme, Literaturverwaltungssoftware)
        \item Physische und soziale Umgebung: Universität und Zuhause, oft in Bibliotheken oder Cafés
    \end{itemize}
    \item \textbf{Anton}:
    \begin{itemize}
        \item Alter: 30 Jahre
        \item Beruf: Fußballprofi
        \item Nutzungsverhalten: Nutzt den \gls{browser}, um den Spielverlauf zu analysieren und aggregierte Statistiken zu erarbeiten
        \item Spezifische Bedürfnisse: Muss \gls{tab}s schnell wiederfinden können, benötigt eine gute Performance und eine übersichtliche Darstellung der \gls{tab}s
        \item Herausforderungen: Langsame Ladezeiten und unübersichtliche \gls{tab}-Verwaltung
        \item Digitale Umgebung: \gls{tab}let und Laptop, häufig genutzte Software: Analyse-Tools, Video-Streaming-Dienste
        \item Physische und soziale Umgebung: Unterwegs und zu Hause, oft in Umkleideräumen und während Reisen
    \end{itemize}
\end{enumerate}

\textbf{Zugänglichkeit der Informationen:}
Die Ergebnisse der Kontextanalyse und die definierten Personas werden dem Gestaltungsteam in Form von detaillierten Berichten und Präsentationen zur Verfügung gestellt, um sicherzustellen, dass sie zu geeigneten Zeitpunkten und in geeigneter Form zugänglich sind.

