\subsection{Verstehen und Spezifizieren des Nutzungskontextes}

\textbf{Ziel:} 
Das Ziel dieser Phase ist es, ein tiefes Verständnis dafür zu entwickeln, wer die Nutzer des Webbrowsers sind, in welchem Kontext sie den Browser verwenden und welche spezifischen Anforderungen und Bedürfnisse sie haben. 
Dies bildet die Grundlage für die Entwicklung benutzerzentrierter Lösungen wie in diesem Fall eine Erweiterung des Browsers zur Verwaltung von Browser-Tabs mit einer Baumstruktur.

Um diese Informationen zu sammeln, wurden verschiedene Methoden eingesetzt, darunter Umfragen und Interviews mit den Nutzern der durch einen Fragebogen gestützt wurde.


\textbf{Fragebogen:}
\begin{itemize}
    \item Wie alt sind Sie?
    \item In welchem Kontext nutzen Sie den Webbrowser? (z.B. Arbeit, Studium, Freizeit)
    \item Wie viele Tabs haben Sie durchschnittlich geöffnet?
    \item Welche Funktionen des Browsers nutzen Sie am häufigsten?
    \item Welche Herausforderungen oder Frustrationen haben Sie bei der Nutzung des Browsers?
    \item Welche Features würden Sie sich im Browser wünschen?
    \item Welche Hardware und andere Software nutzen Sie in Verbindung mit dem Browser?
    \item Wie sieht Ihre physische und soziale Umgebung bei der Nutzung des Browsers aus? (z.B. Arbeitsplatz, Home-Office, unterwegs)
\end{itemize}

\textbf{Bestätigung der Ergebnisse:}
Die gesammelten Informationen wurden durch Feedback-Sitzungen mit den Nutzern validiert.
Daraufhin wurden die Personas definiert welche die Nutzergruppen repräsentieren. 
Dies stellte sicher, dass die Anforderungen und Bedürfnisse korrekt erfasst wurden.

\textbf{Personas:}
\begin{enumerate}
    \item \textbf{Robert}:
    \begin{itemize}
        \item Alter: 35 Jahre
        \item Beruf: Softwareentwickler
        \item Arbeitsumfeld: Arbeitet von zu Hause aus
        \item Nutzungsverhalten: Nutzt den Browser hauptsächlich zur Recherche im Projekt- und Arbeitsumfeld
        \item Spezifische Bedürfnisse: Hat oft über 100 Tabs geöffnet, benötigt eine effiziente Tab-Verwaltung und schnelle Suchfunktionen innerhalb der Tabs
        \item Herausforderungen: Verliert oft den Überblick über geöffnete Tabs
        \item Digitale Umgebung: Nutzt einen leistungsstarken Desktop-Computer mit mehreren Monitoren, häufig verwendete Software: IDEs, Dokumentations-Tools
        \item Physische und soziale Umgebung: Home-Office, arbeitet oft allein, gelegentlich in virtuellen Teamsitzungen
    \end{itemize}
    \item \textbf{Lena}:
    \begin{itemize}
        \item Alter: 25 Jahre
        \item Beruf: Studentin
        \item Nutzungsverhalten: Nutzt den Browser für Recherche und zum Schreiben von Hausarbeiten
        \item Spezifische Bedürfnisse: Nutzt ca. 50 Tabs, benötigt gute Bookmarking-Funktionen und eine intuitive Benutzeroberfläche, die das Organisieren und Wiederfinden von Tabs erleichtert
        \item Herausforderungen: Schwierigkeiten beim Organisieren von Recherchematerial
        \item Digitale Umgebung: Laptop mit Standard-Software (z.B. Textverarbeitungsprogramme, Literaturverwaltungssoftware)
        \item Physische und soziale Umgebung: Universität und Zuhause, oft in Bibliotheken oder Cafés
    \end{itemize}
    \item \textbf{Anton}:
    \begin{itemize}
        \item Alter: 30 Jahre
        \item Beruf: Fußballprofi
        \item Nutzungsverhalten: Nutzt den Browser, um den Spielverlauf zu analysieren und aggregierte Statistiken zu erarbeiten
        \item Spezifische Bedürfnisse: Muss Tabs schnell wiederfinden können, benötigt eine gute Performance und eine übersichtliche Darstellung der Tabs
        \item Herausforderungen: Langsame Ladezeiten und unübersichtliche Tab-Verwaltung
        \item Digitale Umgebung: Tablet und Laptop, häufig genutzte Software: Analyse-Tools, Video-Streaming-Dienste
        \item Physische und soziale Umgebung: Unterwegs und zu Hause, oft in Umkleideräumen und während Reisen
    \end{itemize}
\end{enumerate}

\textbf{Zugänglichkeit der Informationen:}
Die Ergebnisse der Kontextanalyse und Personas werden bei der Implementierung dem Gestaltungsteam in Form von detaillierten Berichten und Präsentationen zur Verfügung gestellt, um sicherzustellen, dass sie zu geeigneten Zeitpunkten und in geeigneter Form zugänglich sind.


