\section{Problembeschreibung und Zielsetzung}
In der modernen digitalen Welt sind \gls{browser} zu unverzichtbaren Werkzeugen für berufliche und private Zwecke geworden. 
Sie ermöglichen es den Nutzern, eine Vielzahl von Informationsquellen und Anwendungen gleichzeitig zu nutzen. 
Insbesondere für Power-User, Entwickler, Forscher und Studierende ist die Nutzung von \gls{browser} und deren \gls{tab}s eine gängige Praxis, um effizient zwischen verschiedenen Aufgaben und Informationsquellen zu wechseln. 
Allerdings führt die zunehmende Anzahl geöffneter \gls{tab}s oft zu Unübersichtlichkeit und ineffizientem Arbeiten.

Aktuelle Browser bieten nur eine lineare Darstellung der \gls{tab}s, was zu mehreren Problemen führt:
\begin{itemize}
    \item \textbf{Unübersichtlichkeit:} Bei vielen geöffneten \gls{tab}s wird die Navigation zunehmend schwieriger.
    \item \textbf{Keine Beziehung zwischen \gls{tab}s:} \gls{tab}s sind unabhängig voneinander und können nicht gruppiert oder hierarchisch organisiert werden.
    \item \textbf{Ineffizienz:} Nutzer verbringen unnötig viel Zeit damit, den richtigen Tab zu finden und zu verwalten.
\end{itemize}
Diese Probleme beeinträchtigen die Produktivität und die Nutzererfahrung erheblich. 
Eine effizientere Lösung zur Verwaltung von \gls{browser}-\gls{tab}s ist daher dringend erforderlich.

\subsection{Zielsetzung}
Das Ziel dieser Arbeit ist es, ein Konzept für die Begleitung des Entwicklungsprozesses hinsichtlich der \ac{UX} zu entwickeln. 
Im Fokus steht die Einführung einer Baumstruktur zur Verwaltung von \gls{browser}-\gls{tab}s. 
Durch diese hierarchische Organisation der \gls{tab}s soll die Übersichtlichkeit und Navigation verbessert werden, um eine positive Nutzererfahrung zu gewährleisten. 
Der \ac{UX}-Anteil des Entwicklungsprozesses wird detailliert geplant, um eine nutzerzentrierte Lösung zu entwickeln.

\subsection{Nicht-Ziele}
Es ist wichtig zu betonen, dass die tatsächliche empirische Durchführung sowie die Einbeziehung realer Testpersonen nicht Bestandteil dieser Arbeit sind. 
Ebenso wird keine tatsächliche Umsetzung des Konzepts in einem realen \gls{browser} angestrebt. 
Der Schwerpunkt liegt auf der theoretischen Planung und der Darstellung des \ac{UX}-Designprozesses.
