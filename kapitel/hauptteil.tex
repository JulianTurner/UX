% Wie analysieren Sie den Kontext?
% Wie soll der UX-Designprozess ablaufen? Wann und auf welche Weise führen Sie Prototypen-Tests durch?
% Wie evaluieren Sie die User Experience der fertigen Anwendung? Welche Arten von
% Nutzertests führen Sie durch und welche empirischen Methoden verwenden Sie?
\section{User-Centered Design}

Die Wahl des \ac{UCD} Ansatzes für die Entwicklung der \gls{browser}-Erweiterung basiert auf der Notwendigkeit, Produkte zu schaffen, die den tatsächlichen Bedürfnissen und Erwartungen der Benutzer entsprechen. 
\ac{UCD} bietet einen strukturierten, iterativen Ansatz zur Entwicklung benutzerzentrierter Lösungen.
Der \ac{UCD}-Ansatz wurde gewählt, um sicherzustellen, dass die Entwicklung der \gls{browser}-Erweiterung eng an den Bedürfnissen der Nutzer ausgerichtet ist. Dies wird durch folgende Faktoren begründet:

\begin{itemize}
    \item \textbf{Fokus auf Benutzerbedürfnisse:} \ac{UCD} stellt die Nutzer in den Mittelpunkt des Entwicklungsprozesses. Dies ermöglicht eine tiefere Einsicht in die spezifischen Anforderungen und Herausforderungen, denen die Benutzer gegenüberstehen, und stellt sicher, dass die entwickelten Lösungen diese Bedürfnisse effektiv adressieren.
    \item \textbf{Iterative Entwicklung:} Durch den iterativen Charakter des \ac{UCD}-Prozesses können Lösungen kontinuierlich verbessert werden. Feedback wird in regelmäßigen Abständen eingeholt und in die nächsten Entwicklungszyklen integriert, was zu einer ständigen Optimierung des Produkts führt.
    \item \textbf{Risikominimierung:} Durch frühzeitige Einbeziehung der Nutzer und regelmäßige Tests mit Prototypen können potenzielle Probleme und Schwachstellen bereits in frühen Phasen des Entwicklungsprozesses identifiziert und behoben werden.
\end{itemize}

\textbf{Anwendung von \ac{UCD} in diesem Projekt:}
In diesem Projekt wurde der \ac{UCD}-Ansatz durch folgende Maßnahmen umgesetzt:

\begin{itemize}
    \item \textbf{Nutzerforschung:} Durchführung von Umfragen, Interviews und Kontextanalysen, um ein tiefes Verständnis der Nutzerbedürfnisse und Nutzungskontexte zu gewinnen.
    \item \textbf{Personas:} Erstellung von detaillierten Personas, die typische Nutzergruppen repräsentieren und deren spezifische Anforderungen und Herausforderungen verdeutlichen.
    \item \textbf{Prototyping und Testing:} Entwicklung von Papierprototypen und \gls{browser}-Erweiterungen, gefolgt von Usability-Tests und Feedback-Sitzungen, um kontinuierliche Verbesserungen zu ermöglichen.
    \item \textbf{Iterative Entwicklung:} Regelmäßige Überprüfung und Anpassung der Lösungen basierend auf dem gesammelten Nutzerfeedback und den Ergebnissen der Usability-Tests.
\end{itemize}

\textbf{Zusammenfassung:}
Die Wahl des Human-Centered Design Ansatzes für die Entwicklung der \gls{browser}-Erweiterung stellt sicher, dass die entwickelten Lösungen eng an den tatsächlichen Bedürfnissen der Benutzer ausgerichtet sind. Durch die iterative und benutzerzentrierte Vorgehensweise wird die Benutzerakzeptanz erhöht, das Risiko von Fehlentwicklungen minimiert und die Wahrscheinlichkeit eines erfolgreichen Produkts deutlich gesteigert.

\subsection{Verstehen und Spezifizieren des Nutzungskontextes}

Das Ziel dieser Phase ist es, ein tiefes Verständnis dafür zu entwickeln, wer die Nutzer des \gls{browser} sind, in welchem Kontext sie den \gls{browser} verwenden und welche spezifischen Anforderungen und Bedürfnisse sie haben. 
Um diese Informationen zu sammeln, wurden verschiedene Methoden eingesetzt, darunter Umfragen und Interviews, die durch einen Fragebogen gestützt wurden.

\textbf{Fragebogen:}
\begin{itemize}
    \item Wie alt sind Sie?
    \item In welchem Kontext nutzen Sie den \gls{browser}? (z.B. Arbeit, Studium, Freizeit)
    \item Wie viele \gls{tab}s haben Sie durchschnittlich geöffnet?
    \item Welche Funktionen des \gls{browser}s nutzen Sie am häufigsten?
    \item Welche Herausforderungen oder Frustrationen haben Sie bei der Nutzung des \gls{browser}s?
    % \item Welche Features würden Sie sich im \gls{browser} wünschen?
    \item Welche Hardware und andere Software nutzen Sie in Verbindung mit dem \gls{browser}?
    \item Wie sieht Ihre physische und soziale Umgebung bei der Nutzung des \gls{browser}s aus? (z.B. Arbeitsplatz, Home-Office, unterwegs)
\end{itemize}

\textbf{Bestätigung der Ergebnisse:}
Die gesammelten Informationen wurden durch Feedback-Sitzungen mit den Nutzern validiert. 
Daraufhin wurden Personas definiert, die die Nutzergruppen repräsentieren. 
Dies stellte sicher, dass die Anforderungen und Bedürfnisse korrekt erfasst wurden.

\textbf{Personas:}
\begin{enumerate}
    \item \textbf{Robert}:
    \begin{itemize}
        \item Alter: 35 Jahre
        \item Beruf: Softwareentwickler
        \item Arbeitsumfeld: Arbeitet von zu Hause aus
        \item Nutzungsverhalten: Nutzt den \gls{browser} hauptsächlich zur Recherche im Projekt- und Arbeitsumfeld
        \item Spezifische Bedürfnisse: Hat oft über 100 \gls{tab}s geöffnet, benötigt eine effiziente \gls{tab}-Verwaltung und schnelle Suchfunktionen innerhalb der \gls{tab}s
        \item Herausforderungen: Verliert oft den Überblick über geöffnete \gls{tab}s
        \item Digitale Umgebung: Nutzt einen leistungsstarken Desktop-Computer mit mehreren Monitoren, häufig verwendete Software: \ac{IDE}, Dokumentations-Tools
        \item Physische und soziale Umgebung: Home-Office, arbeitet oft allein, gelegentlich in virtuellen Teamsitzungen
    \end{itemize}
    \item \textbf{Lena}:
    \begin{itemize}
        \item Alter: 25 Jahre
        \item Beruf: Studentin
        \item Nutzungsverhalten: Nutzt den \gls{browser} für Recherche und zum Schreiben von Hausarbeiten
        \item Spezifische Bedürfnisse: Nutzt ca. 50 \gls{tab}s, benötigt gute Bookmarking-Funktionen und eine intuitive Benutzeroberfläche, die das Organisieren und Wiederfinden von \gls{tab}s erleichtert
        \item Herausforderungen: Schwierigkeiten beim Organisieren von Recherchematerial
        \item Digitale Umgebung: Laptop mit Standard-Software (z.B. Textverarbeitungsprogramme, Literaturverwaltungssoftware)
        \item Physische und soziale Umgebung: Universität und Zuhause, oft in Bibliotheken oder Cafés
    \end{itemize}
    \item \textbf{Anton}:
    \begin{itemize}
        \item Alter: 30 Jahre
        \item Beruf: Fußballprofi
        \item Nutzungsverhalten: Nutzt den \gls{browser}, um den Spielverlauf zu analysieren und aggregierte Statistiken zu erarbeiten
        \item Spezifische Bedürfnisse: Muss \gls{tab}s schnell wiederfinden können, benötigt eine gute Performance und eine übersichtliche Darstellung der \gls{tab}s
        \item Herausforderungen: Langsame Ladezeiten und unübersichtliche \gls{tab}-Verwaltung
        \item Digitale Umgebung: \gls{tab}let und Laptop, häufig genutzte Software: Analyse-Tools, Video-Streaming-Dienste
        \item Physische und soziale Umgebung: Unterwegs und zu Hause, oft in Umkleideräumen und während Reisen
    \end{itemize}
\end{enumerate}

\textbf{Zugänglichkeit der Informationen:}
Die Ergebnisse der Kontextanalyse und die definierten Personas werden dem Gestaltungsteam in Form von detaillierten Berichten und Präsentationen zur Verfügung gestellt, um sicherzustellen, dass sie zu geeigneten Zeitpunkten und in geeigneter Form zugänglich sind.


\subsection{Festlegen der Benutzerbelange und der Erfordernisse}

\textbf{Ziel:} Das Ziel dieses Schrittes ist es, die Benutzerbelange und Erfordernisse für ein \gls{browser}-Plugin zur Tab-Verwaltung mit einer Baumstruktur zu spezifizieren. Diese Anforderungen dienen als Grundlage für das Design und die Entwicklung des Plugins.
\textbf{Benutzerbelange und Erfordernisse:}

\begin{enumerate}
    \item \textbf{Übersichtlichkeit und Organisation:}
    \begin{itemize}
        \item Nutzer müssen in der Lage sein, eine große Anzahl von Tabs effizient zu verwalten.
        \item Eine Baumstruktur soll eine hierarchische Organisation der Tabs ermöglichen.
        \item Tabs sollten einfach in der Baumstruktur verschoben, gruppiert und verschachtelt werden können.
    \end{itemize}
    
    \item \textbf{Benutzerfreundlichkeit:}
    \begin{itemize}
        \item Das Plugin muss eine intuitive und leicht verständliche Benutzeroberfläche haben.
        \item Drag-and-Drop-Funktionalität zum Verschieben und Organisieren von Tabs.
        \item Schneller Zugriff auf häufig genutzte Funktionen und Einstellungen.
    \end{itemize}
    
    \item \textbf{Leistungsfähigkeit:}
    \begin{itemize}
        \item Das Plugin soll auch bei einer großen Anzahl von Tabs flüssig und schnell arbeiten.
        \item Minimale Auswirkungen auf die \gls{browser}leistung.
    \end{itemize}
    
    \item \textbf{Such- und Filterfunktionen:}
    \begin{itemize}
        \item Möglichkeit, Tabs nach Titel, URL oder Inhalt zu durchsuchen.
        \item Filteroptionen, um Tabs nach verschiedenen Kriterien zu sortieren (z.B. nach Erstellungsdatum, Domain).
    \end{itemize}
    
    \item \textbf{Anpassbarkeit:}
    \begin{itemize}
        \item Nutzer sollen die Möglichkeit haben, das Erscheinungsbild und die Funktionen des Plugins an ihre Bedürfnisse anzupassen.
        \item Unterstützung für verschiedene Themes und Layout-Optionen.
    \end{itemize}
    
    \item \textbf{Synchronisation:}
    \begin{itemize}
        \item Möglichkeit zur Synchronisation der Tab-Baumstruktur über mehrere Geräte hinweg.
    \end{itemize}
\end{enumerate}

\textbf{Methoden zur Ermittlung der Anforderungen:}

\begin{itemize}
    \item \textbf{Benutzerbefragungen:} Durchführen von Umfragen und Interviews mit den Nutzern, um ihre spezifischen Bedürfnisse und Wünsche zu erfassen.
    \item \textbf{Wettbewerbsanalyse:} Analyse bestehender Tab-Verwaltungs-Plugins, um deren Stärken und Schwächen zu identifizieren und daraus Anforderungen abzuleiten.
    \item \textbf{Nutzungsszenarien und Use Cases:} Erstellen von detaillierten Nutzungsszenarien und Use Cases, um die verschiedenen Anwendungsfälle des Plugins zu dokumentieren.
    \item \textbf{Prototyping und Benutzer-Feedback:} Entwicklung von Prototypen des Plugins und Einholen von Feedback der Nutzer, um die Anforderungen zu validieren und zu verfeinern.
\end{itemize}

\textbf{Dokumentation der Anforderungen:}

Die Anforderungen werden in einem Anforderungsdokument festgehalten, das die Grundlage für das Design und die Entwicklung des Plugins bildet. Es wird regelmäßig überprüft und aktualisiert, um sicherzustellen, dass es den aktuellen Bedürfnissen der Nutzer entspricht.


\textbf{Ziel:}

% - Leistung im Bezug auf betriebliche & finanzielle Ziele
% - Vorschriften & Gesetze
% - Zusammenarbeit mit anderen Systemen
% - Durchführbarkeit von Betrieb & Wartung (z.B. SLA, Alternativen)

% - Verwenden des Wissens aus Ergonomie, Psychologie, ...
% - Konkretisierung von Gestaltungslösungen mit Pseudoimplementierungen (z.B. Prototypen, Mockups)
% - Wiederholung des Prozesses bis Zufriedenheit



\subsection{Entwerfen von Gestaltungslösungen}
\subsection{Bewerten der Lösungen gegenüber den Anforderungen}
% Ziele:

% - Rückkopplung für die Gestaltung
% - Zielerreichung überprüfen
% - Überwachung der Langzeitnutzung

% Plan für die Bewertung -> Erfolgskriterien, Kennzahlen, Methoden